\addcontentsline{toc}{chapter}{Abstract}

\begin{abstract}

Az egyszerűbb csillagászati számítások is már sok, monoton számítási műveletet igényelnek. Ezek többségében gömbháromszögekben történő geometriai számolások, vagy teljesen empirikus módon definiált összefüggések szükséges kombinációiból állnak. A közeli égbolton látható mozgások (Nap, bolygók, holdak, stb.) látszólagos, vagy valódi, valamint a látható távolabbi objektumok látszó mozgásainak modelljei emiatt könnyen programozható rendszert alkotnak. A beadandó feladatban különböző csillagászati számításokhoz írtam egy terminálban futó, ASCII grafikával rendelkező segédprogramot.

\end{abstract}