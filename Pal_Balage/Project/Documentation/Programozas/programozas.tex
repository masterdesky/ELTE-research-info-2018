\chapter{Programozás}

\label{ch:programozas}

\section{Használt programok}

A teljes programkódot Windows 7, valamint Windows 10 alatt írtam, legnagyobb része VS Code-ban és VS 2015 Community-ban született, és teljesen nulláról lett megírva. Több logikai funkció teszteléséhez is Jupyter Notebookot használtam. Mivel két gépen is programoztam - egyetemen laptopon, otthon asztali gépen - ezért nagyon aktívan hasznát vettem a GitHub-nak is, hogy folyamatosan push-oljam és pull-oljam a projektet a két gép között. Ezt az Anaconda mellé érkező, Git-es MinGW terminálból valósítottam meg, de volt több alkalom, hogy GitHub Desktop-ot használtam.\\
Ezt a LaTeX file-t a Sharelatex új bétaverziójában, az Overleaf v2-ben írtam meg (csupán kíváncsiságból, próba gyanánt).\\
A C++ file-t Clang, valamint CMake segítségével fordítottam le.

\section{A megírás folyamata és a program szerkezete}

A programot először Python 3.6.5 alatt írtam meg, amit hosszú munkával be is fejeztem. Ezt as kódot először a \href{https://github.com/masterdesky/Csillesz-Calculations-KUTINF}{Csillesz-Calculations-KUTINF} nevű GitHub repository-m alatt fejlesztettem.\\
Miután ez minden tervezett funcionalitással rendelkezett, elkezdtem a programot átfordítani C++ nyelvre. Ezt közben Clang-al teszteltem és ellenőriztem. \\ 
Valamivel a C++-re történő fordítás elkezdése után migráltam a teljes repó tartalmát az eredeti csoportom, \href{https://github.com/masterdesky/Kutinfo_gyakorlat_KA_LM_PB}{Kutinfo\_gyakorlat\_KA\_LM\_PB} nevű GitHub repository-jába. A további munkákat és frissítéseket itt folytattam.\\
Ebből a verzióból az ábrázolások idő hiányában már sajnos kimaradtak, azok csak a Python verzióban vannak meg. Miután a C++ verzió is elkészült - az ábrázolásokat leszámítva - azt CMake-el fordítottam, majd teszteltem le.\\ \\
A Python verzió jelenlegi, elkészült formájában 4000, míg a C++ verzió valamivel több, mint 6500 soros lett. Mindkét verzióban kihasználtam a nyelv által kínált lehetőségeket, mint pl Pythonban a dict-eket és az azokkal végezhető műveleteket, C++-ben pedig pl. az std::map vagy a try{} except{} struktúrákat és jópár különféle könyvtárat is.


\section{Alkalmazás}

A két program a következő funkciókkal rendelkezik:
\begin{itemize}
  \item Csillagászati (Égi-) koordinátarendszerek közti átváltás
  \item Távolságszámítás a Földön két tetszőleges pont között
  \item Csillagidő számítása tetszőleges helyen és tetszőleges helyi időben
  \item Napnyugta/Napkelte, polgári-, navigációs-, és csillagászati szürkület, valamint delelés, éjfél és csillagászati éjszaka hosszának és idejének számítása tetszőleges helyen, és időben
  \item Tetszőleges gömbháromszög hiányzó adatainak kiszámítása az adottak alapján
  \item A "Csillagászati észlelési gyakorlatok II" című tárgy év végi beadandójának teljes megírása egy gombnyomásra
\end{itemize}
\pagebreak
Kizárólag a Python-os verzió rendelkezik az alábbiakkal:
\begin{itemize}
  \item Napóra árnyékának felrajzolása az óraszög függvényében a tavaszi- és őszi napéjegyenlőség, a nyári- valamint téli napforduló, valamint opcionálisan egy tetszőleges választott napon az év során, tetszőlegesen választott évben és helyen
  \item A Nap analemmájának ábrázolása tetszőleges helyen és évben
\end{itemize}
A program a Csillagászati észlelési gyakorlatok órán megtanultaknál jóval pontosabb adatokat ad eredményül a különféle számítások során. Több interneten is elérhető komoly szoftverrel, valamint tényleges előrejelzéssel összehasonlítottam az eredményeit, és a pontossága legtöbb esetben 1 percnél is pontosabb.